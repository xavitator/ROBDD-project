\documentclass[a4paper, oneside]{report}
\usepackage{amsfonts,amsmath,amssymb}
\usepackage[utf8]{inputenc}
\usepackage[francais]{babel}
\usepackage{graphicx}
\usepackage{polynom}
\usepackage[T1]{fontenc}
\usepackage{mathenv}
\usepackage{array}
\usepackage{mdwtab}
\usepackage[colorlinks=true,linkcolor=black]{hyperref}
\frenchbsetup{StandardLists=true}

\begin{document}


\title{Diagramme de décision binaire - ROBDD \\ Rapport pour le projet Mathématique et Informatique de L3 de 2018-2019  }
\date{\today}
\author{Sébastien Palmer et Xavier Durand \\~\\ Encadré par Sedki Boughattas }
\maketitle

\tableofcontents{}

\newpage

%%%%% INTRODUCTION %%%%%%%
\chapter*{Introduction}
\addcontentsline{toc}{chapter}{Introduction}

\section*{Présentation du sujet}
\section*{Qu'est ce qu'une expression propositionnelle}
\section*{Qu'est ce qu'une ROBDD (présentation rapide)}
\section*{Problématique}
\section*{plan}


%%%%% Chapitre 1 %%%%%%%
\chapter{Représentation sous ROBDD}

\section{Système complet de $\varphi$}
\section{démo de l'existence}
\section{démo de l'unicité}
\section{A quoi correspond le graphe, et rapport avec $\varphi$}
\section{Représentation correcte -> pas de perte d'informations}

%%%%% Chapitre 2 %%%%%%%
\chapter{Construction d'une ROBDD}

\section{présentation de l'algorithme}
\section{Présentation de l'algo}
\section{exemple d'exécution de l'algo}
\section{Etude des complexités}

%%%%% Chapitre 3 %%%%%%%
\chapter{Intérêt et optimisation}

\section{Raison de la ROBDD}
\section{Utilisation possible comme Sat-solveur mais mauvais}
\section{Théorie sur les ordres}

%%%%% Conclusion %%%%%%%
\chapter*{Conclusion}
\addcontentsline{toc}{chapter}{Conclusion}

\section*{Nouvelle approche d'une expression propositionnelle}
\section*{Optimisation que cela apporte en fonction de l'ordre}
\section*{Représentation simple d'une expression prop}
\section*{Théorie développée et approche de recherche pour les ordres}
\section*{Ce que le projet nous a apporté}

%%%%% Bibliographie %%%%%%%
\chapter*{Bibliographie}
\addcontentsline{toc}{chapter}{Bibliographie}
Ajouter les différents articles sur lesquels on s'est basé.

\end{document}